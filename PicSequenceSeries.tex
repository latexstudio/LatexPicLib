% !TEX encoding = UTF-8 Unicode
% !TEX TS-program = xelatex
\chapter{數列級數圖形}

\begin{tikzpicture}[scale=0.7]

\newcommand{\drawCutedPoint}[3]{
    
    \foreach \v in {0,1,...,#3}
    {
        \coordinate (cCutedPt) at ($ (#1) !{\v/#3}! (#2) $);
        \draw[ultra thick,fill] (cCutedPt) circle (1.5pt);
    }
    
}

\newcommand{\drawHexagon}[1]{
    \coordinate (cAPt) at (0: #1);
    \coordinate (cBPt) at (60:#1);
    \coordinate (cCPt) at (120:#1);
    \coordinate (cDPt) at (180:#1);
    \coordinate (cEPt) at (240:#1);
    \coordinate (cFPt) at (300:#1);
    
    \draw (cAPt)--(cBPt)--(cBPt)--(cCPt)--(cDPt)--(cEPt)--(cFPt) --cycle;
    
    \drawCutedPoint{cAPt}{cBPt}{#1};
    \drawCutedPoint{cBPt}{cCPt}{#1};
    \drawCutedPoint{cCPt}{cDPt}{#1};
    \drawCutedPoint{cDPt}{cEPt}{#1};
    \drawCutedPoint{cEPt}{cFPt}{#1};
    \drawCutedPoint{cFPt}{cAPt}{#1};
}

\begin{scope}[shift={(0,0)}]
\foreach \t in {1}
{
    \begin{scope}[shift={(120:\t)}]
    \drawHexagon{\t}
    \end{scope}
}
\node at (-0.5,-1){圖一};
\end{scope}
\begin{scope}[shift={(5,0)}]
\foreach \t in {1,2}
{
    \begin{scope}[shift={(120:\t)}]
    \drawHexagon{\t}
    \end{scope}
}
\node at (-1,-1){圖二};
\end{scope}
\begin{scope}[shift={(12,0)}]
\foreach \t in {1,2,...,3}
{
    \begin{scope}[shift={(120:\t)}]
    \drawHexagon{\t}
    \end{scope}
}
\node at (-1.5,-1){圖三};
\end{scope}
\end{tikzpicture}

\begin{tikzpicture}[scale=0.7]

\foreach \beeCol in {0,1,2,3}
{
    \begin{scope} [shift = {(0:3 *\beeCol)} ]
    \begin{scope} [shift = {(0,0)}]
    \foreach \y in {0,1,2,3}
    {
        \begin{scope}[shift={(270:1.73* \y)}]
        %描述極坐標點
        \coordinate (cAPt) at (0: 1);
        \coordinate (cBPt) at (60:1);
        \coordinate (cCPt) at (120:1);
        \coordinate (cDPt) at (180:1);
        \coordinate (cEPt) at (240:1);
        \coordinate (cFPt) at (300:1);
        
        \draw (cAPt)--(cBPt)--(cBPt)--(cCPt)--(cDPt)--(cEPt)--(cFPt) --cycle;
        \end{scope}
    }
    \foreach \y in {0,1,2,3}
    {
        \begin{scope}[shift={(330:1.73)}]
        \begin{scope}[shift={(270:1.73* \y)}]
        %描述極坐標點
        \coordinate (cAPt) at (0: 1);
        \coordinate (cBPt) at (60:1);
        \coordinate (cCPt) at (120:1);
        \coordinate (cDPt) at (180:1);
        \coordinate (cEPt) at (240:1);
        \coordinate (cFPt) at (300:1);
        
        \draw (cAPt)--(cBPt)--(cBPt)--(cCPt)--(cDPt)--(cEPt)--(cFPt) --cycle;
        \end{scope}
        \end{scope}
        
    }
    
    \end{scope}
    \end{scope}
}
\end{tikzpicture}

        \begin{tikzpicture}
\newcommand\drawitemtwoone[1]
{        
    \begin{scope}[shift={(#1,0)}]
    {
        \filldraw[fill=lightgray,draw=black] (0,0) rectangle (2,1);
        \draw[dashed] (1,0) -- (1,1);
    }
    \end{scope}
}
\begin{scope}[shift={(0,0)}]


\foreach \x in {0,...,11}
{
    \begin{scope}[shift={(\x, 0)}]
    \filldraw[fill=white,draw=black] (0,0) rectangle (1,1);
    \end{scope}
}

\foreach \x in {1,3,6,8,10}
{
    \drawitemtwoone{\x};
}

\end{scope}
\node[below] at (6,0){圖一};
\end{tikzpicture}

\begin{tikzpicture}
\newcommand\drawitemtwoone[1]
{        
    \begin{scope}[shift={(#1,0)}]
    {
        \filldraw[fill=lightgray,draw=black] (0,0) rectangle (2,1);
        \draw[dashed] (1,0) -- (1,1);
    }
    \end{scope}
}
\begin{scope}[shift={(0,0)}]


\foreach \x in {0,...,11}
{
    \begin{scope}[shift={(\x, 0)}]
    \filldraw[fill=white,draw=black] (0,0) rectangle (1,1);
    \end{scope}
}

\foreach \x in {0,2,4,7,9}
{
    \drawitemtwoone{\x};
}

\end{scope}
\node[below] at (6,0){圖二};
\end{tikzpicture}

\begin{tikzpicture}
\newcommand\drawitemtwoone[1]
{        
    \begin{scope}[shift={(#1,0)}]
    {
        \filldraw[fill=lightgray,draw=black] (0,0) rectangle (2,1);
        \draw[dashed] (1,0) -- (1,1);
    }
    \end{scope}
}
\begin{scope}[shift={(0,0)}]


\foreach \x in {0,...,11}
{
    \begin{scope}[shift={(\x, 0)}]
    \filldraw[fill=white,draw=black] (0,0) rectangle (1,1);
    \end{scope}
}

\foreach \x in {1,4,6}
{
    \drawitemtwoone{\x};
}

\end{scope}
\node[below] at (6,0){圖三};
\end{tikzpicture}
\newpage
\begin{tikzpicture}
\begin{scope} 
\foreach \y in {0}
{
    \begin{scope}[shift={(330:1.73*2)}]
    \begin{scope}[shift={(270:1.73* \y)}]
    %描述極坐標點
    \coordinate (cAPt) at (0: 1);
    \coordinate (cBPt) at (60:1);
    \coordinate (cCPt) at (120:1);
    \coordinate (cDPt) at (180:1);
    \coordinate (cEPt) at (240:1);
    \coordinate (cFPt) at (300:1);
    
    \draw (cAPt)--(cBPt)--(cBPt)--(cCPt)--(cDPt)--(cEPt)--(cFPt) --cycle;
    \end{scope}
    \end{scope}
}

\end{scope}
\node[above] at (1,0) {圖一};
\node[above] at (5,0) {圖二};
\node[above] at (10,0) {圖三};

\begin{scope} [shift = {(5,0)}]
\foreach \y in {0,1}
{
    \begin{scope}[shift={(330:1.73)}]
    \begin{scope}[shift={(270:1.73* \y)}]
    %描述極坐標點
    \coordinate (cAPt) at (0: 1);
    \coordinate (cBPt) at (60:1);
    \coordinate (cCPt) at (120:1);
    \coordinate (cDPt) at (180:1);
    \coordinate (cEPt) at (240:1);
    \coordinate (cFPt) at (300:1);
    
    \draw (cAPt)--(cBPt)--(cBPt)--(cCPt)--(cDPt)--(cEPt)--(cFPt) --cycle;
    \end{scope}
    \end{scope}
    
}

\foreach \y in {0}
{
    \begin{scope}[shift={(330:1.73*2)}]
    \begin{scope}[shift={(270:1.73* \y)}]
    %描述極坐標點
    \coordinate (cAPt) at (0: 1);
    \coordinate (cBPt) at (60:1);
    \coordinate (cCPt) at (120:1);
    \coordinate (cDPt) at (180:1);
    \coordinate (cEPt) at (240:1);
    \coordinate (cFPt) at (300:1);
    
    \draw (cAPt)--(cBPt)--(cBPt)--(cCPt)--(cDPt)--(cEPt)--(cFPt) --cycle;
    \end{scope}
    \end{scope}
}
\end{scope}

\begin{scope} [shift = {(12,0)}]
\foreach \y in {0,1,2}
{
    \begin{scope}[shift={(270:1.73* \y)}]
    %描述極坐標點
    \coordinate (cAPt) at (0: 1);
    \coordinate (cBPt) at (60:1);
    \coordinate (cCPt) at (120:1);
    \coordinate (cDPt) at (180:1);
    \coordinate (cEPt) at (240:1);
    \coordinate (cFPt) at (300:1);
    
    \draw (cAPt)--(cBPt)--(cBPt)--(cCPt)--(cDPt)--(cEPt)--(cFPt) --cycle;
    \end{scope}
}
\foreach \y in {0,1}
{
    \begin{scope}[shift={(330:1.73)}]
    \begin{scope}[shift={(270:1.73* \y)}]
    %描述極坐標點
    \coordinate (cAPt) at (0: 1);
    \coordinate (cBPt) at (60:1);
    \coordinate (cCPt) at (120:1);
    \coordinate (cDPt) at (180:1);
    \coordinate (cEPt) at (240:1);
    \coordinate (cFPt) at (300:1);
    
    \draw (cAPt)--(cBPt)--(cBPt)--(cCPt)--(cDPt)--(cEPt)--(cFPt) --cycle;
    \end{scope}
    \end{scope}
    
}

\foreach \y in {0}
{
    \begin{scope}[shift={(330:1.73*2)}]
    \begin{scope}[shift={(270:1.73* \y)}]
    %描述極坐標點
    \coordinate (cAPt) at (0: 1);
    \coordinate (cBPt) at (60:1);
    \coordinate (cCPt) at (120:1);
    \coordinate (cDPt) at (180:1);
    \coordinate (cEPt) at (240:1);
    \coordinate (cFPt) at (300:1);
    
    \draw (cAPt)--(cBPt)--(cBPt)--(cCPt)--(cDPt)--(cEPt)--(cFPt) --cycle;
    \end{scope}
    \end{scope}
}
\end{scope}


\end{tikzpicture}


\newpage

\begin{tikzpicture}[scale=0.7]
%x y axes
\draw[-{Stealth[scale=1.3,angle'=45]},semithick] (-5,0) -- (5,0) node[] {$x$};
\draw[-{Stealth[scale=1.3,angle'=45]},semithick] (0,-4)->(0,5)   node[above] {$y$};
%標示直角角度
\coordinate (cOPt) at (0,0);
\coordinate (cA1Pt) at (0:1);
\coordinate (cA2Pt) at ( $(120:2) + (cA1Pt)$ ) ;
\coordinate (cA3Pt) at ( $(240:3) + (cA2Pt)$ ) ;
\coordinate (cA4Pt) at ( $(0:4) + (cA3Pt)$ ) ;
\coordinate (cA5Pt) at ( $(120:5) + (cA4Pt)$ ) ;
\coordinate (cA6Pt) at ( $(240:6) + (cA5Pt)$ ) ;
\coordinate (cA7Pt) at ( $(0:7) + (cA6Pt)$ ) ;

\draw [ultra thick] (cA1Pt) -- (cA2Pt) -- (cA3Pt) --(cA4Pt) --(cA5Pt) --(cA6Pt) --(cA7Pt)  ;
\small
\foreach \v/\u/\t in 
{cOPt/225/$O$,
    cA1Pt/-10/$A_1$,
    cA2Pt/45/$A_2$,
    cA3Pt/225/$A_3$,
    cA4Pt/-10/$A_4$,
    cA5Pt/45/$A_5$,
    cA6Pt/180/$A_6$,
    cA7Pt/-1/$A_7$
}
{
    \draw[ultra thick,fill] (\v) circle (2pt);
    \node[label={[label distance=-0.25cm]\u:\t}] at (\v){};
    
};
\end{tikzpicture}

    \begin{tikzpicture}[scale=0.6]
%火柴棒排三角形
\coordinate (cOPt) at (0,0);
\coordinate (cAPt) at (0:2);
\coordinate (cBPt) at (60:2);
\draw[ultra thick] (cOPt) -- (cAPt) -- (cBPt) -- (cOPt) circle;

\foreach \v in {cOPt,cAPt,cBPt}
{
    \draw[ultra thick,fill] (\v) circle (3pt);
};

\node[label=270:圖1] at (1,0){};


\begin{scope}[shift={(5,0)}]
\coordinate (cOPt) at (0,0);
\coordinate (cAPt) at (0:2);
\coordinate (cA2Pt) at (0:4);
\coordinate (cBPt) at (60:2);
\coordinate (cB2Pt) at (60:4);
\coordinate (cMPt) at ($ (cA2Pt) !.5! (cB2Pt) $);
\draw[ultra thick] (cOPt) -- (cAPt) -- (cBPt) -- (cOPt) circle;
\draw[ultra thick] (cOPt) -- (cA2Pt) -- (cB2Pt) -- (cOPt) circle;
\draw[ultra thick] (cAPt) --  (cMPt) circle; 

\foreach \v in {cOPt,cAPt,cA2Pt,cBPt,cB2Pt,cMPt}
{
    \draw[ultra thick,fill] (\v) circle (3pt);
};
\node[label=270:圖2] at (2,0){};
\end{scope}

\begin{scope}[shift={(12,0)}]
\coordinate (cOPt) at (0,0);
\coordinate (cAPt) at (0:2);
\coordinate (cA2Pt) at (0:4);
\coordinate (cBPt) at (60:2);
\coordinate (cB2Pt) at (60:4);
\draw[ultra thick] (cOPt) -- (cAPt) -- (cBPt) -- (cOPt) circle;
\draw[ultra thick] (cOPt) -- (cA2Pt) -- (cB2Pt) -- (cOPt) circle;
\draw[ultra thick] (cAPt) -- ($ (cA2Pt) !.5! (cB2Pt) $) circle; 
\begin{scope}[shift={(6,0)}]
\coordinate (cA3Pt) at (0,0);
\coordinate (cCPt) at (-2,0);
\coordinate (cDPt) at (120:2);
\coordinate (cD2Pt) at (120:4);
\coordinate (cD3Pt) at (120:6);

\end{scope}

\draw[ultra thick] (cOPt) -- (cA3Pt) -- (cD3Pt)  -- (cOPt) ;
\coordinate (cMPt) at  ($ (cA2Pt) !.5! (cB2Pt) $);   %中點定位法
\draw[ultra thick] (cMPt) -- (cD2Pt) ;
\draw[ultra thick] (cA2Pt) -- (cDPt);

\foreach \v in {cOPt,cAPt,cA2Pt,cA3Pt,cBPt,cB2Pt,cMPt,cDPt,cD2Pt,cD3Pt}
{
    \draw[ultra thick,fill] (\v) circle (3pt);
};
\node[label=270:圖3] at (3,0){};
\end{scope}
\end{tikzpicture}