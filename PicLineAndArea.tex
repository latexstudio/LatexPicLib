% !TEX encoding = UTF-8 Unicode
% !TEX TS-program = xelatex
\chapter{直線、區域與線性規劃}

    \begin{tikzpicture}
\makeatletter
\tikzset{
    nomorepostaction/.code=\makeatletter\let\tikz@postactions\pgfutil@empty, % From http://tex.stackexchange.com/questions/3184/applying-a-postaction-to-every-path-in-tikz/5354#5354
    my axis/.style={
        postaction={
            decoration={
                markings,
                mark=at position 1 with {
                    \arrow[ultra thick]{latex}
                }
            },
            decorate,
            nomorepostaction
        },
        thin,
        -, % switch off other arrow tips
        every path/.append style=my axis % this is necessary so it works both with "axis lines=left" and "axis lines=middle"
    }
}
\makeatother

\pgfplotsset{ticks=none}
\begin{axis}[
unit vector ratio*=1 1 1,
axis lines = middle,
axis line style = {my axis},
width=11cm,
xlabel={$x$},
ylabel={$y$},
ymax =10
]
%Below the red parabola is defined
\addplot [
domain=-6:6, 
samples=100, 
color=black,ultra thick
]
{+0.5*x+1};
%\addlegendentry{$\$}
%Here the blue parabloa is defined
\addplot [
domain=-6:6, 
samples=100, 
color=black,ultra thick
]
{-2*x+6};
\addplot [
domain=-6:6, 
samples=100, 
color=black,ultra thick
]
{-0.5*x-1};
\node[label=150:$L_1$] at (axis cs:6,4){};
\node[label=230:$L_2$] at (axis cs:-2,10){};
\node[label=70:$L_3$] at (axis cs:-6,2){};
\end{axis}
\end{tikzpicture}

\begin{tikzpicture}
\coordinate (cBPt) at (5,0);
\coordinate (cAPt) at (-2,0) ;

\coordinate (cPPt) at  ($ (cAPt) !0.66! (cBPt) $); 
\draw (cAPt) -- (cBPt) ;
\draw [semithick] (cAPt) -- (cPPt) node [midway, above ] {$\textc{3}$};	

\draw [semithick] (cBPt) -- (cPPt) node [midway, above ] {$\textc{2}$};	


\foreach \v/\u/\t in 
{cAPt/90/$A$,
    cBPt/90/$B$,
    cPPt/90/$P$,
    cAPt/270/$-2$,
    cBPt/270/$13$,
    cPPt/270/$x$
}
{
    \draw[ultra thick,fill] (\v) circle (1pt);
    \node[label=\u:\t] at (\v){};
};
\end{tikzpicture}

\begin{tikzpicture}
\coordinate (cBPt) at (0,0);
\coordinate (cAPt) at (-2,0) ;
\coordinate (cPPt) at  ($ (cAPt) !3! (cBPt) $); 
\draw (cAPt) -- (cBPt) ;
\draw (cAPt) -- (cPPt) ;
\draw [semithick] (cAPt)  .. controls +(30:1cm) and +(150:1cm) .. (cPPt) node [midway, above ] {$\textc{3}$};	

\draw [semithick] (cBPt) .. controls +(-30:1cm) and +(-150:1cm) .. (cPPt) node [midway, below ] {$\textc{2}$};	


\foreach \v/\u/\t in 
{cAPt/90/$A$,
    cBPt/90/$B$,
    cPPt/90/$P$,
    cAPt/270/$-2$,
    cBPt/270/$13$,
    cPPt/270/$x$
}
{
    \draw[ultra thick,fill] (\v) circle (1pt);
    \node[label=\u:\t] at (\v){};
};
\end{tikzpicture}
       
\begin{tikzpicture}
\coordinate (cLineL) at (-4, 0);
\coordinate (cLineR) at (6, 0);
\coordinate (cAL) at (-3,0);
\coordinate (cAR) at (5,0);

\draw  (cLineL) -- (cLineR);
\draw[ultra thick]  (cAL) -- (cAR);

\foreach \v in 
{   0,1,2
}
{
    \draw[dashed] (\v-3,2-0.8*\v) -- (\v-3,0);
    \draw[dashed] (\v+3,2-0.8*\v) -- (\v+3,0);
    
    \begin{scope}[shift={(0,2-0.8*\v)}]
    \def\Bradius{3}
    \coordinate (cBC) at (\v, 0);
    \coordinate (cBR) at (\v + \Bradius, 0);
    \coordinate (cBL) at (\v - \Bradius, 0);
    \draw [-{Stealth[scale=1.3,angle'=45]},semithick]  (cBC) -- (cBR);
    \draw [-{Stealth[scale=1.3,angle'=45]},semithick]  (cBC) -- (cBL);
    \draw[ultra thick,fill] (cBC) circle (0.8pt);
    \node[label=90:$\v$] at (cBC){};
    \node[label=180:$A$] at (cBL){};
    \end{scope}
};

\foreach \v/\u/\t in 
{   cAL/270/$-3$,
    cAR/270/$5$,
    cLineL/180/$B$
}
{
    \draw[ultra thick,fill] (\v) circle (0.8pt);
    \node[label=\u:\t] at (\v){};
};

\end{tikzpicture}

	\begin{tikzpicture}
\coordinate (cAPt) at (0.67,3);
\coordinate (cBPt) at (2,1);
\coordinate (cCPt) at (3.5,2.5);
\coordinate (cDPt) at (1.6,4.4);
%繪製斜線,陰影區 線性規劃的方法
\draw[pattern=north west lines, ultra thick] (cAPt) -- (cBPt)--(cCPt)--(cDPt) -- (cAPt) circle ;

\foreach \v/\u/\t in 
{cAPt/90/$A$,
    cBPt/270/$B$,
    cCPt/0/$C$,
    cDPt/90/$D$,
    cOPt/225/$O$}
{
    \draw[ultra thick,fill] (\v) circle (2pt);
    \node[label=\u:\t] at (\v){};
};			
% x軸 y 軸 axes
\draw[-{Stealth[scale=1.3,angle'=45]},semithick] (-0.5,0) -- (6,0) node[] {$x$};
\draw[-{Stealth[scale=1.3,angle'=45]},semithick] (0,-0.5)->(0,6)   node[above] {$y$};


\end{tikzpicture}

\begin{tikzpicture}[scale=0.7]
\coordinate (cAPt) at (2,3);
\coordinate (cBPt) at (3.95,0);
\coordinate (cCPt) at (0,4);
\coordinate (cDPt) at (0,0);
\coordinate (cPPt) at  (4.34, -0.6);
\coordinate (cQPt) at = (-0.52, 6.88);
\coordinate (cRPt) at = (-0.6, 4.3);
\coordinate (cSPt) at = (9.14, -0.57);
%繪製斜線,陰影區 線性規劃的方法
\draw[pattern=north west lines, ultra thick] (cAPt) -- (cBPt)--(cDPt)--(cCPt) -- (cAPt) circle ;

\draw[semithick] (cQPt) -- (cPPt) node[] {$L_1$};
\draw[semithick] (cRPt) -- (cSPt) node[] {$L_2$};


\draw[ultra thick,fill] (cAPt) circle (2pt);
\node[above ] at (cAPt){$A(2,3)$};			
% x軸 y 軸 axes
\draw[-{Stealth[scale=1.3,angle'=45]},semithick] (-0.5,0) -- (10,0) node[] {$x$};
\draw[-{Stealth[scale=1.3,angle'=45]},semithick] (0,-0.5)->(0,7)   node[above] {$y$};


\end{tikzpicture}